% Hey, Emacs!  This is a -*- mode: latex -*- file!

 \enlargethispage{1.5cm}
 \scriptsize
 \vspace*{-1cm}
 \hspace*{-1.5cm}%
 \ExplSyntaxOn
 \raisebox{-\height+0.7\baselineskip}{%
   \begin{minipage}[t]{.6\textwidth}%\vspace{0pt}%
     \__letgut_orig_includegraphics*[width=\linewidth]{logo-gut.pdf}
   \end{minipage}%
 }%
 \ExplSyntaxOff
 \hfill%
 \begin{minipage}[t]{.5\textwidth}%\vspace{0pt}%
   \footnotesize\raggedleft%
   Association \gut{}\\
   15 rue des Halles -- \textsc{bp} 74\\
   75001 Paris\\
   France\\
   \nolinkurl{secretariat[at]gutenberg-asso[dot]fr}
 \end{minipage}%

\begin{description}
\item[Site Internet :] \url{https://gutenberg-asso.fr/}
\item[\Cahiers{} :] \url{https://cahiers.gutenberg-asso.fr/} et
  \url{https://www.numdam.org/journals/CG/}
\item[\lettre{} :] \url{https://lettre.gutenberg-asso.fr/}
\item[Problèmes \TeX{}niques :]
  \leavevmode
  \begin{description}
  \item[liste d'entraide :]
    \url{https://gutenberg-asso.fr/-Listes-de-diffusion-}
  \item[site de questions et réponses :]
    \url{https://texnique.fr/}
  \item[foire aux questions :]
    \url{https://faq.gutenberg-asso.fr/}
  \end{description}
\end{description}

\alertbox{%
  \raggedright%
  Cette association est la vôtre : faites-nous part de
  vos idées, de vos envies, de vos préoccupations
  à l'adresse \nolinkurl{secretariat@gutenberg-asso.fr}.

  Adhérents, vous pouvez aussi échanger sur la vie de l’association sur la liste
  de diffusion \nolinkurl{adherents@gutenberg-asso.fr}.%
}

\title{Adhésion à l'association}
\label{letgut_label_adhesions}
\scriptsize%
\alertbox{%
  \raggedright%
  \gut{} étant reconnue d’intérêt général, vous recevrez en temps voulu un
  justificatif vous permettant de bénéficier d’une réduction fiscale de
  66~\% du montant de votre cotisation ou de votre don.%
}
\begin{itemize}
\item % Adhésions et abonnements
  Les adhésions sont à renouveler  en début d'année
  pour l'année civile.
  % \item Il n'y a pas de lettre de rappel, chaque membre
  %   doit faire son renouvellement annuel; %\hspace*{1em}
  %   merci de renvoyer spontanément le bulletin ci-dessous
  %   en début d'année.
\item Les administrations peuvent joindre un bon de commande
  revêtu de la signature de la personne responsable ;
  les étudiants doivent joindre un justificatif.
  % \item Si vous souhaitez que vos coordonnées restent
  %   confidentielles, merci de le signaler.
\end{itemize}

\section[Tarifs \the\year]{Tarifs\footnote{Dans ce tableau, une personne
  physique à tarif réduit est étudiant, demandeur d’emploi ou plus largement
  toute personne non redevable de l’impôt sur le revenu (sur présentation d’un
  justificatif) ; un organisme peut être doté ou non de la personnalité
  morale : laboratoire de recherche public, etc.} \the\year}

Les membres de \gut\ peuvent adhérer à l'association internationale, le
\acf{tug}, et recevoir son bulletin \tugboat{} à un tarif
préférentiel\footnote{En tarif normal, 65~€ (au lieu de 85~\$) ; en tarif
  étudiant, 40~€ (au lieu de 55~\$).} :

\begin{center}
  \begin{tblr}{lr@{ }l}
    \toprule
    \textbf{Type d'adhésion}                                 & \textbf{Prix} &                   \\
    \midrule
    Personne physique                                        & 30~€          &                   \\[-0.5 mm]
    Personne physique + adhésion \textsc{tug}                & 95~€          & ($=$ 30~€ + 65~€) \\[-0.5 mm]
    Personne physique à tarif réduit                         & 15~€          &                   \\[-0.5 mm]
    Personne physique à tarif réduit + adhésion \textsc{tug} & 55~€          & ($=$ 15~€ + 40~€) \\[-0.5 mm]
    Association d'étudiants                                  & 65~€          &                   \\[-0.5 mm]
    Organisme ou association à but non lucratif              & 130~€         &                   \\[-0.5 mm]
    Personne morale à but lucratif                           & 229~€         &                   \\[-0.5 mm]
    \bottomrule
  \end{tblr}
\end{center}

 \section{Règlements}

Les règlements peuvent s'effectuer par :
\begin{itemize}
% \item virement bancaire%
%   \footnote{L'\textsc{iban} de l'association
%   est à demander au secrétariat.}
% %  FR76 1870 7000 3003 0191 3568 475)% banque pop
% %  FR76 3000 3001 0900 0372 6086 280)% socgen Gap, compte courant
% \leavevmode
%
%   \alertbox{%
%     Veillez à bien \emph{indiquer vos nom et prénom}
%     dans les références du virement !%
%   }
\item paiement en ligne sécurisé\footnote{En carte
  bancaire.} :
  \url{https://gutenberg-asso.fr/Adherer-en-ligne}
\item bulletin et chèque :
  \url{https://gutenberg-asso.fr/Adherer-a-l-association}
\end{itemize}

 \vfill

\begin{center}
  \begin{tblr}{
      width=.8\linewidth,
      colspec={
        % < (Just not to disturb the parentheses
        % balancing detection of my editor.)
        >{\microtypesetup{protrusion=false}\bfseries}r@{ }X
      },
      vline{1,3}={solid},
      hline{1,7}={solid},
      rowsep=0mm,
      row{1} = {abovesep+=.25cm},
      row{2} = {belowsep+=.125cm},
      row{6} = {belowsep+=.25cm},
      % colsep=2.5mm,
    }
    \SetCell[c=2]{c} \emph{La \lettregut}                                 \\
    \SetCell[c=2]{c} \mdseries Bulletin irrégulomestriel \& apériodique
    de l'association \gut                                                 \\
    Directeur de la publication :
                                      & \person{Bideault, P.}        \\
    Comité de rédaction :
                                      & {P. Bideault, D. Bitouzé,
      C. Chevalier \& B. Dumont}                       \\
    Adresse de la rédaction :         &
    {
      Association \gut\\
      15 rue des Halles -- \textsc{bp} 74 -- 75001 Paris
    }
    \\
    \acs{issn} : & \letgutissn
  \end{tblr}
\end{center}
\vfill
\mbox{}
\clearpage
